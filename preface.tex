%%%%%%%%%%%%%%%%%%%%%%foreword.tex%%%%%%%%%%%%%%%%%%%%%%%%%%%
% sample foreword
%
% Use this file as a template for your own input.
%
%%%%%%%%%%%%%%%%%%%%%%%% Springer %%%%%%%%%%%%%%%%%%%%%%%%%%

\preface

Due to its power to solve large scale problems and analyse vast amounts of data that would be difficult or time consuming for humans to deal with manually, the field of Computational Intelligence has grown tremendously in importance over the past years. We nowadays see widespread use of computational intelligence in the most varied applications. A few examples include approaches to detect credit card fraud, recognise faces, transcribe voice to text, identify spam, route and schedule deliveries, design aerodynamic high speed trains, etc. 

It is thus not surprising that we see a growing number of people who are keen to learn about this field. However, there is a lack of open resources that combine several different types of computational intelligence approaches in one place, so that people can easily get an introduction to this field. Those eager to learn about computational intelligence may also struggle to get help from others when trying to understand existing approaches, whereas those willing to start teaching this topic may struggle to find free resources to guide them. 

This \textit{open} book has been created as a community effort to overcome these issues. The notion of \textit{openness} of this book includes, but goes beyond open access. In addition to being available through an open license so that resources on computational intelligence are accessible to all, this book is hosted in github at:

\

\noindent \url{https://github.com/ieee-cis/IEEE-CIS-Open-Access-Book-Volume-1}. 

\

Such initiative will enable the book to be continuously improved over time through pull requests to fix typos, add clarifications, add new exercises, add examples of open software code, add video lectures on the content, etc. Therefore, this book is \textit{open} for the community to propose enhancements over time. The book is also associated to github discussion boards, so that people can ask questions and the community can help with answering those questions, creating an \textit{open} community that all can join in. 

If you would like to propose an enhancement through a pull request to this book, we ask you to first contact the current chair of the IEEE Computational Intelligence Society Education Portal Subcommittee (\url{https://cis.ieee.org/}). The chair will advise you on how to proceed. Minor changes to existing chapters will be handled by the subcommittee directly, whereas the subcommittee will liaise with the original authors to obtain their consent for incorporating larger pull requests. 

We thank all the authors who have contributed chapters to this book, all the anonymous reviewers who have reviewed the chapters, and all the community members who will contribute with this book in the future.

We hope that you will find the book a useful resource to learn about computational intelligence.

\vspace{\baselineskip}
\begin{flushright}\noindent
Leandro L. Minku, on behalf of the editors of the\\
IEEE CIS Computational Intelligence Open Book -- First Edition 
\end{flushright}


