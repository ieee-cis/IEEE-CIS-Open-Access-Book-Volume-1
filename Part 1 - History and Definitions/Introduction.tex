\title{Introduction}
\author{Leandro L. Minku}
\institute{University of Birmingham, UK}
\maketitle
\label{chp:history}
\label{chp:definitions}


Even though the term \textit{Computational Intelligence} (CI) has been used for many years, no single agreed definition exists so far. Traditionally, CI has been considered to be the ``theory, design, application and development of biologically and linguistically motivated computational paradigms" \cite{ciswebsite}. Possibly, this definition has been proposed because many researchers adopting the term CI were associated to the IEEE Computational Intelligence Society (CIS), which has roots in the IEEE Neural Networks Society (NNS) and its predecessor the IEEE Neural Networks Council (NNC). 

As explained in \cite{cishistory}, the IEEE Neural Networks Council (NNC), established in November 1989, was the publisher of the IEEE Transactions on Neural Networks, the IEEE Transactions on Evolutionary Computation, and the IEEE Transactions on Fuzzy Systems journals. Their field of interest was specified as ``the theory, design, application, and development of biologically and linguistically motivated computational paradigms'' \cite{NNSEditorial}, which precisely matches the CI definition given above. The IEEE NNC then transitioned to become the IEEE NNS in November 2001, with a view of continuing to focus on the same field of interest \cite{NNSEditorial}. In November 2003, the IEEE NNS then changed its name to IEEE CIS. 

As evidenced by the three journals that were originally published by the IEEE NNC, the three main pillars of CI have traditionally been neural networks, evolutionary computation, and fuzzy systems \cite{ciswebsite}. These are biologically- and linguistically-inspired topics, being well aligned with the CI definition above. However, many different biologically-inspired algorithms have been proposed since then. Moreover, key events sponsored or technically co-sponsored by the IEEE CIS nowadays also include other CI topics that are not necessarily biologically- or linguistically-inspired. This includes flagship conferences such as the International Joint Conference on Neural Networks (IJCNN) and the IEEE Congress on Evolutionary Computation (IEEE CEC). Therefore, the definition above does not include all topics that may currently be referred to as CI topics.
%For instance, IJCNN not only welcomes submissions on the topic of neural networks (originally biologically-inspired), but also other general machine learning approaches to learn from data (not necessarily biologically-inspired). IEEE CEC not only welcomes submissions on evolutionary computation (originally biologically inspired), but also on other general approaches such as meta-heuristics to search for solutions to optimisation problems (not necessarily biologically-inspired). The topics of machine learning and meta-heuristics will be introduced in Parts II and III of this open book.

As of October 2022, the Wikipedia entry of Computational Intelligence \cite{ciswiki} explains that the ``expression computational intelligence (CI) usually refers to the ability of a computer to learn a specific task from data or experimental observation.'' This definition is fairly general, being well aligned with topics covered in some of the key conferences in the field and not requiring biological or linguistic inspiration. In particular, this definition is inclusive but not limited to the traditional CI pillars of neural networks and evolutionary computation. However, it arguably does not match very well the traditional pillar of fuzzy systems. This is because fuzzy systems do not necessarily learn from data or experimental observation. Instead, they may be based on linguistically-inspired knowledge bases provided by humans. 

According to Bezdek \cite{Bezdek}, it is believed that the term CI originated from the name of the Canadian's Artificial Intelligence (AI) society founded in 1974: the Canadian Society for Computational Studies of Intelligence. Nick Cercone and Gordon McCalla, who were members of the society, decided to create an AI journal. After much debate and some influence from the term ``Computer Vision", they decided that the term CI was more adequate to describe their field than the term AI \cite{Bezdek}. Therefore, they created the journal named International Journal of Computational Intelligence (IJCI) in 1983. It is unclear from Bezdek \cite{Bezdek}'s account whether this means that the term CI was originally intended to be a sub-field of AI. However, this is a possible way to view the field of CI.

The term AI itself has many different possible definitions \cite{RussellNorvig}. One of the well received definitions is the one explained by Russell and Norvig \cite{RussellNorvig}: AI is concerned with the study of agents that act rationally, i.e., agents that act so as to achieve the best (expected) outcome. This is a very general definition that can be seen as incorporating all of the CI approaches. It includes the three traditional CI pillars as well as other non-biologically-inspired and non-linguistically-inspired approaches commonly seen in current CI venues. It includes CI approaches that learn to perform a task based on data or experimental observation, and those that perform a task based on linguistically-inspired approaches such as fuzzy systems.  It also includes approaches that are not usually considered as CI approaches, such as agents that act rationally based on a knowledge base consisting of crisp logical statements (i.e., statements using boolean logics such as propositional logic, first order logic, etc). Therefore, \textit{one may consider CI to be AI algorithms that are not based on crisp logical statements}. This is the view that this book will take. The focus will be on the algorithms, rather than on the systems where they may be embedded.

The next parts of this book will give several examples of CI algorithms that fit within this definition. By the end of this book, we hope readers to become more familiar with the field, gaining a better understanding of what this definition means and entails. However, as  pointed out by Bezdek \cite{Bezdek}, many different definitions of CI and arguments exist \cite{CIDef1,CIDef2,CIDef3,CIDef4,CIDef5,CIDef6,CIDef7,ciswebsite}. Therefore, even though this book is adopting this definition, it is important to emphasize that this is not a universal definition.


\bibliographystyle{unsrt}
\bibliography{bibliography}
