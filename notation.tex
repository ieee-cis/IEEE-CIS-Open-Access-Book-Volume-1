\notation

In general, the following mathematical notations will be used in this book:

\begin{itemize}
\item Scalar: lower case, e.g., $a$, $b$.
\item Column vector: lower case, bold, e.g., $\textbf{x}$.
\item Vector element: lower case with subscript, e.g., $x_1$, $x_2$.
\item If enumerating vectors (e.g., having multiple vectors), superscript will be used to differentiate this from indices, e.g., $\textbf{x}^{(1)}$, $\textbf{x}^{(2)}$.
\item Matrix: upper case, bold, e.g., $\textbf{X}$.
\item Matrix element: upper case with subscripts, e.g., $X_{1,2}$.
\item When considering that a matrix is a vector of vectors, row $i$ and column $j$ can be represented by $\textbf{x}^{(i)}_j$.
\item Sets: calligraphy font in upper case, e.g., $\mathcal{T}$.
\item Generic data structure with unspecified format (e.g., it could be a vector, a matrix, or any other structure): lower case, bold, e.g., $\textbf{x}$.
\end{itemize}